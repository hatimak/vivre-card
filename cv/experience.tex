\begin{cvexperience}

\cvExperienceTitle{EN}{M.\,Sc. Thesis ``Field-Programmable Gate Array based Real-Time Control and Simulation''} % Job title
\cvExperienceTitle{DE}{Masterarbeit ``Field-Programmable Gate Array basierte Echtzeitregelung und -simulation''} % Job title

\cvExperienceOrganisation{EN}{Institute of Energy and Climate Research (IEK-10), Forschungszentrum Juelich GmbH} % Organization
\cvExperienceOrganisation{DE}{Institut f{\"u}r Energie- und Klimaforschung (IEK-10) am Forschungszentrum J{\"u}lich GmbH} % Organization

\cvExperienceSubtitle{EN}{Supervisors: Univ.-Prof. Dr.-Ing. Andrea Benigni, Steffen Vogel, M.\,Sc., Dr.-Ing. Lukas Razik} % Subtitle
%% TODO @hatimak: German Subtitles
\cvExperienceSubtitle{DE}{Supervisors: Univ.-Prof. Dr.-Ing. Andrea Benigni, Steffen Vogel, M.\,Sc., Dr.-Ing. Lukas Razik} % Subtitle

\cvExperienceLocation{EN}{Aachen, Germany \ifbool{showEmojis} {\acvHeaderIconSep\emoji{flag-germany}} {} } % Location
\cvExperienceLocation{DE}{Aachen, Deutschland \ifbool{showEmojis} {\acvHeaderIconSep\emoji{flag-germany}} {} } % Location

\cvExperienceDates{EN}{Apr. 2021 -- Dec. 2021} % Date(s)
\cvExperienceDates{DE}{Apr. 2021 -- Dez. 2021} % Date(s)

\cvExperienceDescription{EN}{%
	\begin{cvitems}
		\item {Developed design using soft-core microprocessors to rapidly prototype control-loop algorithms for FPGA-based real-time simulators of power systems and to allow independent formulation of power system and control-loop models.}
		\item {Introduced \textit{control} and \textit{data-logger} soft-cores, each based on MicroBlaze soft-core microprocessor from Xilinx and implemented on Xilinx Virtex Ultrascale+ VCU118 board, respectively dedicated to running control algorithms at switching frequency and logging simulation data at each time step.} 
		\item {Conceptualised heterogenous architecture using multiple and dedicated soft-core microprocessors, enabling hierarchical control-loop systems and fine-grained administration of real-time simulation.}
  		\item {Assembled final work product using combination of proprietary Xilinx IPs from Vivado IP Integrator, HLS modules of power systems generated from ORTiS, self-authored Verilog RTL modules and binaries for soft-core microprocessor using Xilinx SDK.}
	\end{cvitems}%
}
%% TODO @hatimak: German descriptions
\cvExperienceDescription{DE}{%
	\begin{cvitems}
		\item {Developed design using soft-core microprocessors to rapidly prototype control-loop algorithms for FPGA-based real-time simulators of power systems and to allow independent formulation of power system and control-loop models.}
		\item {Introduced \textit{control} and \textit{data-logger} soft-cores, each based on MicroBlaze soft-core microprocessor from Xilinx and implemented on Xilinx Virtex Ultrascale+ VCU118 board, respectively dedicated to running control algorithms at switching frequency and logging simulation data at each time step.} 
		\item {Konzipierte eine heterogene Architektur mit mehreren speziellen Soft-Core-Mikroprozessoren, die hierarchische Regelkreissysteme und eine detailliertere Verwaltung der Echtzeitsimulation erm{\"o}glicht.}
  		\item {Assembled final work product using combination of proprietary Xilinx IPs from Vivado IP Integrator, HLS modules of power systems generated from ORTiS, self-authored Verilog RTL modules and binaries for soft-core microprocessor using Xilinx SDK.}
	\end{cvitems}%
}

\cvExperienceFlush
%---------------------------------------------------------

\cvExperienceTitle{EN}{Intern} % Job title
\cvExperienceTitle{DE}{Praktikant} % Job title

\cvExperienceOrganisation{EN}{Institute of Energy and Climate Research (IEK-10), Forschungszentrum Juelich GmbH} % Organization
\cvExperienceOrganisation{DE}{Institut f{\"u}r Energie- und Klimaforschung (IEK-10) am Forschungszentrum J{\"u}lich GmbH} % Organization

\cvExperienceSubtitle{EN}{Supervisor: Dr.-Ing. Lukas Razik, Head, HPC Department, IEK-10} % Subtitle
\cvExperienceSubtitle{DE}{Supervisor: Dr.-Ing. Lukas Razik, Head, HPC Department, IEK-10} % Subtitle

\cvExperienceLocation{EN}{Remote \ifbool{showEmojis} {\acvHeaderIconSep\emoji{world-map}} {} } % Location
\cvExperienceLocation{DE}{Home Office \ifbool{showEmojis} {\acvHeaderIconSep\emoji{world-map}} {} } % Location

\cvExperienceDates{EN}{Oct. 2020 -- Feb. 2021} % Date(s)
\cvExperienceDates{DE}{Okt. 2020 -- Feb. 2021} % Date(s)

\cvExperienceDescription{EN}{%
	\begin{cvitems}
		\item {Implemented power sytems models using open-source code-generation tool ORTiS targeted toward High-Level Synthesis for RTL co-simulation and real-time simulation on an FPGA.}
		\item {Extended HLS models with memory-mapped AXI4 register interface. Verified hardware models on Xilinx Virtex-7 VC707 FPGA board using remote debugging.}
  		\item {Developed Makefile pipeline on Linux for ORTiS code generation, Vivado High-Level Synthesis, Vivado IP Integrator and FPGA bitstream generation stages.}
	\end{cvitems}%
}
%% TODO @hatimak: German descriptions
\cvExperienceDescription{DE}{%
	\begin{cvitems}
		\item {Implementierte Stromnetzmodelle mit dem Open-Source-Code-Generierungstool ORTiS, das auf High-Level-Synthese f{\"u}r RTL-Co-simulation und Echtzeitsimulation auf einem FPGA ausgerichtet war.}
		\item {Erweiterte HLS-Modelle mit memory-mapped AXI4-Register-Interface. Verifizierte Hardware-Modelle auf Xilinx Virtex-7 VC707 FPGA-Board mit Remote-Debugging.}
  		\item {Entwickelte Makefile-Pipeline unter Linux für ORTiS-Code-Generierung, Vivado High-Level Synthesis, Vivado IP Integrator und FPGA-Bitstream-Generierung.}
	\end{cvitems}%
}

\cvExperienceFlush
%---------------------------------------------------------

\cvExperienceTitle{EN}{Student Assistant} % Job title
\cvExperienceTitle{DE}{Studentische Hilfskraft} % Job title

\cvExperienceOrganisation{EN}{Institute for Automation of Complex Power Systems, E.ON Energy Research Centre} % Organization
\cvExperienceOrganisation{DE}{Institute for Automation of Complex Power Systems, E.ON Energy Research Centre} % Organization

\cvExperienceSubtitle{EN}{Supervisor: \href{https://www.acs.eonerc.rwth-aachen.de/go/id/msul}{Steffen Vogel, M.\, Sc.}} % Subtitle
\cvExperienceSubtitle{DE}{Supervisor: \href{https://www.acs.eonerc.rwth-aachen.de/go/id/msul}{Steffen Vogel, M.\, Sc.}} % Subtitle

\cvExperienceLocation{EN}{Aachen, Germany \ifbool{showEmojis} {\acvHeaderIconSep\emoji{flag-germany}} {} } % Location
\cvExperienceLocation{DE}{Aachen, Deutschland \ifbool{showEmojis} {\acvHeaderIconSep\emoji{flag-germany}} {} } % Location

\cvExperienceDates{EN}{May 2019 -- Sept. 2020} % Date(s)
\cvExperienceDates{DE}{Mai 2019 -- Sept. 2020} % Date(s)

\cvExperienceDescription{EN}{%
	\begin{cvitems}
		\item {Integrated Xilinx FPGA boards into \href{https://villas.fein-aachen.org}{VILLAS} co-simulation platform by designing an architecture built on top of Aurora 8B/10B serial protocol.}
%		\item {Engineered Tcl-Makefile system of scripts to automate design generation and bitstream compilation. Developers could utilise system to collaborate on version-agnostic local toolchains and significantly reduce commit and check-out sizes.}
		\item {Engineered Tcl-Makefile system of scripts to automate design generation and bitstream compilation.}
		\item {Developed bare-metal driver programs in C/C++ for FPGA firmware.}
	\end{cvitems}%
}
\cvExperienceDescription{DE}{%
	\begin{cvitems}
		\item {Integrierte Xilinx-FPGA-Boards in die VILLAS-Cosimulations-Plattform durch den Aufbau einer Architektur auf dem seriellen Aurora 8B/10B-Protokoll.}
%		\item {Entwickelte ein Tcl-Makefile-System mit Skripten zur Automatisierung der Design-Generierung und Bitstream-Kompilierung. Entwickler k{\"o}nnten das System nutzen, um mit versionsunabhängigen lokalen Toolchains zu arbeiten und die Commit- und Checkout-Gr{\"o}{\ss}en erheblich zu reduzieren.}
		\item {Entwickelte ein Tcl-Makefile-System mit Skripten zur Automatisierung der Design-Generierung und Bitstream-Kompilierung.}
		\item {Entwickelte Bare-Metal-Driver-Programme in C/C++ f{\"u}r FPGA-Firmware.}
	\end{cvitems}%
}

\cvExperienceFlush
%---------------------------------------------------------

\cvExperienceTitle{EN}{Senior Research Fellow} % Position
\cvExperienceTitle{DE}{Senior Research Fellow} % Position

\cvExperienceOrganisation{EN}{``Underwater Target Motion Analysis with Passive Sensors'',\newline\href{https://www.drdo.gov.in/labs-and-establishments/naval-physical-oceanographic-laboratory-npol}{Naval Physical \& Oceanographic Laboratory} (DRDO), Ministry of Defence, Govt of India} % Title
\cvExperienceOrganisation{DE}{``Underwater Target Motion Analysis with Passive Sensors'',\newline\href{https://www.drdo.gov.in/labs-and-establishments/naval-physical-oceanographic-laboratory-npol}{Naval Physical \& Oceanographic Laboratory} (DRDO), Ministry of Defence, Govt of India} % Title

\cvExperienceSubtitle{EN}{Supervisor: \href{http://www.tutorialpoint.org/ShovanBhaumik/index.html}{Dr Shovan Bhaumik}, Sponsor: \href{https://www.drdo.gov.in/labs-and-establishments/naval-physical-oceanographic-laboratory-npol}{Naval Physical \& Oceanographic Laboratory (DRDO)}} % Subtitle
\cvExperienceSubtitle{DE}{Supervisor: \href{http://www.tutorialpoint.org/ShovanBhaumik/index.html}{Dr Shovan Bhaumik}, Sponsor: \href{https://www.drdo.gov.in/labs-and-establishments/naval-physical-oceanographic-laboratory-npol}{Naval Physical \& Oceanographic Laboratory (DRDO)}} % Subtitle

\cvExperienceLocation{EN}{\href{https://www.iitp.ac.in}{IIT Patna}, India \ifbool{showEmojis} {\acvHeaderIconSep\emoji{flag-india}} {} } % Location
\cvExperienceLocation{DE}{\href{https://www.iitp.ac.in}{IIT Patna}, Indien \ifbool{showEmojis} {\acvHeaderIconSep\emoji{flag-india}} {} } % Location

\cvExperienceDates{EN}{May 2018 -- Nov. 2018} % Date(s)
\cvExperienceDates{DE}{Mai 2018 -- Nov. 2018} % Date(s)

\cvExperienceDescription{EN}{%
	\begin{cvitems} % Description(s)
%		\item {Implemented advanced tracking filters in MATLAB, like Distributed Extended Kalman Filter, Shifted Rayleigh FilterSequential Monte Carlo methods, for the Bearings-only Tracking problem.}
		\item {Implemented advanced tracking filters in MATLAB for the Bearings-only Tracking problem.}
		\item {Simulated performance of modern filters on real field manoeuvre data from Indian Navy, and prepared comparative study.}
		\item{Concluded that Shifted Rayleigh Filter outperforms other filters in terms of computational complexity while still being superior at tracking target.}
	\end{cvitems}%
}
%% TODO @hatimak: German descriptions
\cvExperienceDescription{DE}{%
	\begin{cvitems} % Description(s)
%		\item {Implementierte fortgeschrittene Tracking-Filters in MATLAB, wie Distributed Extended Kalman Filter, Shifted Rayleigh Filter und Sequenzielle Monte-Carlo-Methode, f{\"u}r das Bearings-only Tracking Problem.}
		\item {Implementierte fortgeschrittene Tracking-Filters in MATLAB f{\"u}r das Bearings-only Tracking-Problem.}
		\item {Simulierte die Leistung von modernen Filtern anhand realer Manöverdaten der indischen Marine und erstellte eine vergleichende Studie.}
		\item{Concluded that Shifted Rayleigh Filter outperforms other filters in terms of computational complexity while still being superior at tracking target.}
	\end{cvitems}%
}

\cvExperienceFlush
%---------------------------------------------------------

\cvExperienceTitle{EN}{B.\,Tech. Thesis \href{https://github.com/hatimak/sigma}{``Hardware Architecture of a Family of Sigma-Point Kalman Filters for Bayesian Estimation''}} % Position
\cvExperienceTitle{DE}{Bachelorarbeit \href{https://github.com/hatimak/sigma}{``Hardware Architecture of a Family of Sigma-Point Kalman Filters for Bayesian Estimation''}} % Position

\cvExperienceOrganisation{EN}{Control and Instrumentation Lab} % Title
\cvExperienceOrganisation{DE}{Control and Instrumentation Lab} % Title

\cvExperienceSubtitle{EN}{Supervisor: \href{http://www.tutorialpoint.org/ShovanBhaumik/index.html}{Dr Shovan Bhaumik}, Assistant Professor, IIT Patna} % Subtitle
\cvExperienceSubtitle{DE}{Supervisor: \href{http://www.tutorialpoint.org/ShovanBhaumik/index.html}{Dr Shovan Bhaumik}, Assistant Professor, IIT Patna} % Subtitle

\cvExperienceLocation{EN}{\href{https://www.iitp.ac.in}{IIT Patna}, India \ifbool{showEmojis} {\acvHeaderIconSep\emoji{flag-india}} {} } % Location
\cvExperienceLocation{DE}{\href{https://www.iitp.ac.in}{IIT Patna}, Indien \ifbool{showEmojis} {\acvHeaderIconSep\emoji{flag-india}} {} } % Location

\cvExperienceDates{EN}{Aug. 2017 -- May 2018} % Date(s)
\cvExperienceDates{DE}{Aug. 2017 -- Mai 2018} % Date(s)

\cvExperienceDescription{EN}{%
	\begin{cvitems} % Description(s)
		\item {Designed and implemented a parallel architecture of Sigma-point Kalman filtering algorithms on an FPGA by independently conceptualised parallel routine for Cholesky decomposition in $\mathcal{O} \left( N \right)$ time complexity.}
		\item {Further optimised resource usage of parallel Cholesky decomposition architecture for maximum processor utilisation to achieve $\mathcal{O} \left( \frac{1}{4} N^2 \right)$ resource complexity, as compared to $\mathcal{O} \left( \frac{1}{2} N^2 \right)$ resource complexity of state-of-the-art.}
%		\item {Customised and implemented a matrix inversion routine to compute covariance inverses in $\mathcal{O} \left( N \right)$ time complexity.}
		\item {Implemented parallel architectures using Verilog HDL and Xilinx Vivado on Xilinx Zynq-7000 ZC702 and Digilent Nexys4 DDR FPGA boards, making use of open-source floating-point IPs and Xilinx Vivado IPs.}
		\item {Presented final work product to the professors of the department and was one of only two students to receive unanimous 10 / 10 grade from cohort of 50 candidates. Nominated for Best B.\,Tech. Thesis award from Dept of Electrical Engineering.}
	\end{cvitems}%
}
%% TODO @hatimak: German descriptions
\cvExperienceDescription{DE}{%
	\begin{cvitems} % Description(s)
		\item {Designed and implemented a parallel architecture of Sigma-point Kalman filtering algorithms on an FPGA by independently conceptualised parallel routine for Cholesky decomposition in $\mathcal{O} \left( N \right)$ time complexity.}
		\item {Further optimised resource usage of parallel Cholesky decomposition architecture for maximum processor utilisation to achieve $\mathcal{O} \left( \frac{1}{4} N^2 \right)$ resource complexity, as compared to $\mathcal{O} \left( \frac{1}{2} N^2 \right)$ resource complexity of state-of-the-art.}
%		\item {Customised and implemented a matrix inversion routine to compute covariance inverses in $\mathcal{O} \left( N \right)$ time complexity.}
		\item {Implemented parallel architectures using Verilog HDL and Xilinx Vivado on Xilinx Zynq-7000 ZC702 and Digilent Nexys4 DDR FPGA boards, making use of open-source floating-point IPs and Xilinx Vivado IPs.}
		\item {Presented final work product to the professors of the department and was one of only two students to receive unanimous 10 / 10 grade from cohort of 50 candidates. Nominated for Best B.\,Tech. Thesis award from Dept of Electrical Engineering.}
	\end{cvitems}%
}

\cvExperienceFlush
%---------------------------------------------------------

\cvExperienceTitle{EN}{Student Developer} % Title
\cvExperienceTitle{DE}{Studentischer Softwareentwickler} % Title

\cvExperienceOrganisation{EN}{\href{https://fossi-foundation.org}{Free and Open Source Silicon Foundation}, \href{https://github.com/librecores/gsoc-museum-edsac}{``EDSAC Museum on FPGA''}} % Organization
\cvExperienceOrganisation{DE}{\href{https://fossi-foundation.org}{Free and Open Source Silicon Foundation}, \href{https://github.com/librecores/gsoc-museum-edsac}{``EDSAC Museum on FPGA''}} % Organization

\cvExperienceSubtitle{EN}{Mentor: \href{http://www.jeremybennett.com}{Dr Jeremy Bennett}, \href{https://www.embecosm.com/about/meet-the-team/jeremy-bennett/}{Founder \& Chief Executive}, Embecosm Ltd} % Subtitle
\cvExperienceSubtitle{DE}{Mentor: \href{http://www.jeremybennett.com}{Dr Jeremy Bennett}, \href{https://www.embecosm.com/about/meet-the-team/jeremy-bennett/}{Founder \& Chief Executive}, Embecosm Ltd} % Subtitle

\cvExperienceLocation{EN}{\href{https://summerofcode.withgoogle.com/archive/2017/projects/6470218444439552/}{Google Summer of Code 2017}} % Location
\cvExperienceLocation{DE}{\href{https://summerofcode.withgoogle.com/archive/2017/projects/6470218444439552/}{Google Summer of Code 2017}} % Location

\cvExperienceDates{EN}{May 2017 -- Aug. 2017} % Date(s)
\cvExperienceDates{DE}{Mai 2017 -- Aug. 2017} % Date(s)

\cvExperienceDescription{EN}{%
	\begin{cvitems} % Description(s)
		\item {Built Verilog model of historic EDSAC computer from original but incomplete documentation in collaboration with members of ``The EDSAC Replica Project'' team (TNMOC, Bletchley Park, UK).}
		\item {Programmed and simulated EDSAC architecture and ISA on \href{https://mystorm.uk/}{myStorm} Lattice iCE FPGA board using open-source toolchains, like \href{http://www.clifford.at/yosys/}{Yosys} and \href{http://iverilog.icarus.com}{iverilog}.}
%		\item {Designed and implemented a modified-UART communication protocol to support external I/O interfaces to the system.}
		\item {Coordinated with team of younger students in UK to build hardware imitation of EDSAC memory delay line, teleprinter and paper tape reader.}
		\item {Demonstrated final work product at \href{http://chiphack.org/chiphack-2017.html}{ChipHack 2017} workshop and \href{https://youtu.be/EZkJOyOcYiY}{presented} at \href{https://orconf.org/2017/\#edsac}{ORConf 2017 digital design conference} in Hebden Bridge, UK, for which full sponsorship was received.}
	\end{cvitems}%
}
%% TODO @hatimak: German descriptions
\cvExperienceDescription{DE}{%
	\begin{cvitems} % Description(s)
		\item {Built Verilog model of historic EDSAC computer from original but incomplete documentation in collaboration with members of ``The EDSAC Replica Project'' team (TNMOC, Bletchley Park, UK).}
		\item {Programmed and simulated EDSAC architecture and ISA on \href{https://mystorm.uk/}{myStorm} Lattice iCE FPGA board using open-source toolchains, like \href{http://www.clifford.at/yosys/}{Yosys} and \href{http://iverilog.icarus.com}{iverilog}.}
%		\item {Designed and implemented a modified-UART communication protocol to support external I/O interfaces to the system.}
		\item {Coordinated with team of younger students in UK to build hardware imitation of EDSAC memory delay line, teleprinter and paper tape reader.}
		\item {Demonstrated final work product at \href{http://chiphack.org/chiphack-2017.html}{ChipHack 2017} workshop and \href{https://youtu.be/EZkJOyOcYiY}{presented} at \href{https://orconf.org/2017/\#edsac}{ORConf 2017 digital design conference} in Hebden Bridge, UK, for which full sponsorship was received.}
	\end{cvitems}%
}

\cvExperienceFlush
%---------------------------------------------------------

\cvExperienceTitle{EN}{Student Developer} % Title
\cvExperienceTitle{DE}{Studentischer Softwareentwickler} % Title

\cvExperienceOrganisation{EN}{\href{https://www.coreboot.org}{Coreboot} (\href{https://www.flashrom.org/Flashrom}{Flashrom}), \href{https://drive.google.com/drive/u/1/folders/0B-Cccp-WWmeuUlh5M3pxT0cyQm8}{``Read/Write Multiple Status Registers and Lock/Unlock Memory on SPI Chips''}} % Organization
\cvExperienceOrganisation{DE}{\href{https://www.coreboot.org}{Coreboot} (\href{https://www.flashrom.org/Flashrom}{Flashrom}), \href{https://drive.google.com/drive/u/1/folders/0B-Cccp-WWmeuUlh5M3pxT0cyQm8}{``Read/Write Multiple Status Registers and Lock/Unlock Memory on SPI Chips''}} % Organization

\cvExperienceSubtitle{EN}{} % Subtitle
\cvExperienceSubtitle{DE}{} % Subtitle

\cvExperienceLocation{EN}{\href{https://summerofcode.withgoogle.com/archive/2016/projects/5439533130711040/}{Google Summer of Code 2016}} % Location
\cvExperienceLocation{DE}{\href{https://summerofcode.withgoogle.com/archive/2016/projects/5439533130711040/}{Google Summer of Code 2016}} % Location

\cvExperienceDates{EN}{Feb. 2016 -- Aug. 2016} % Date(s)
\cvExperienceDates{DE}{Feb. 2016 -- Aug. 2016} % Date(s)

\cvExperienceDescription{EN}{%
	\begin{cvitems} % Description(s)
		\item {Designed unified abstraction of multiple status registers in SPI Flash-memory chips to provide consistent interface between flashrom and diverse chip manufacturers.}
		\item {Programmed routines to lock/unlock memory space governed by status registers, handle configuration bits, access/lock OTP memory areas, and automatically generate memory protection maps.}
		\item {Developed CLI to expose these features, and tested infrastructure using Raspberry Pi and Teensy development board.}
		\item {Liaised with Sales Executive of GigaDevie from China to arrange for engineering samples and add support for GigaDevice SPI chips.}
	\end{cvitems}%
}
%% TODO @hatimak: German descriptions
\cvExperienceDescription{DE}{%
	\begin{cvitems} % Description(s)
		\item {Designed unified abstraction of multiple status registers in SPI Flash-memory chips to provide consistent interface between flashrom and diverse chip manufacturers.}
		\item {Programmed routines to lock/unlock memory space governed by status registers, handle configuration bits, access/lock OTP memory areas, and automatically generate memory protection maps.}
		\item {Developed CLI to expose these features, and tested infrastructure using Raspberry Pi and Teensy development board.}
		\item {Liaised with Sales Executive of GigaDevie from China to arrange for engineering samples and add support for GigaDevice SPI chips.}
	\end{cvitems}%
}

\cvExperienceFlush
%---------------------------------------------------------

% \cvposition
% 	{Summer Intern} % Job title
% 	{Aficionado Ventures} % Organization
% 	{} % Subtitle
% 	{New Delhi, India} % Location
% 	{May 2016 -- July 2016} % Date(s)
% 	{\begin{cvitems} % Description(s) of tasks/responsibilities
% 		\item {Aficionado was a startup helping restaurants procure quality produce from competitive collection of vendors.}
% 		\item {Designed reactive backend architecture - MeteorJS server, MongoDB database, Heroku/mLab web hosting, and Cordova for cross-platform mobile apps.}
% 		\item {Implemented MVP-stage marketplace platform from scratch with bilingual search for products in English/Hinglish.}
% 		\item {Worked with business team to conduct market research on restaurants and vendors.}
% 	\end{cvitems}}
% %---------------------------------------------------------

\end{cvexperience}
