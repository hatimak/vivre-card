\cvsection{Experience}

\begin{cvexperience}

\cvposition
	{M.Sc. Thesis ``Field-Programmable Gate Array based Real-Time Control and Simulation''} % Job title
	{IEK-10, Forschungszentrum J\"ulich and Institute for Automation of Complex Power Systems} % Organization
	{Supervisors: Univ.-Prof. Dr.-Ing. Andrea Benigni, Steffen Vogel, M.\, Sc., Dr.-Ing. Lukas Razik} % Subtitle
	{Aachen, Germany} % Location
	{Apr. 2021 -- Dec. 2021} % Date(s)
	{\begin{cvitems}
		\item {}
	\end{cvitems}}
%---------------------------------------------------------

\cvposition
	{Student Intern} % Job title
	{IEK-10, Forschungszentrum J\"ulich} % Organization
	{Supervisor: Dr.-Ing. Lukas Razik, Head, HPC Department, IEK-10} % Subtitle
	{J\"ulich, Germany (Remote)} % Location
	{Oct. 2020 -- Feb. 2021} % Date(s)
	{\begin{cvitems}
		\item {Implemented power system models using the open-source code-generation tool ORTiS targeted towards High-Level Synthesis for real-time simulation on an FPGA. Extended the HLS models with a memory-mapped AXI4 register interface.}
  		\item {Developed a Makefile pipeline on Linux to streamline prototyping workflow handling the ORTiS code generation, Vivado High-Level Synthesis, Vivado IP Integrator and FPGA bitstream generation stages.}
%		\item {\# something about remote debugging setup}
	\end{cvitems}}
%---------------------------------------------------------

\cvposition
	{Student Assistant} % Job title
	{Institute for Automation of Complex Power Systems, E.ON Energy Research Centre} % Organization
	{Supervisor: \href{https://www.acs.eonerc.rwth-aachen.de/go/id/msul}{Steffen Vogel, M.\, Sc.}} % Subtitle
	{Aachen, Germany} % Location
	{May 2019 -- Sept. 2020} % Date(s)
	{\begin{cvitems}
		\item {Integrated Xilinx FPGA development boards into the \href{https://villas.fein-aachen.org}{VILLAS} co-simulation platform by designing an architecture built on top of the Aurora 8B/10B serial protocol from Xilinx, thereby providing a consistent interface between the CPU, over PCIe, and the real-time simulators, over a physical fibre link.}
		\item {Engineered a Tcl-Makefile system of scripts to simplify file management of multiple source types and binary files and streamline project management using revision control systems. Developers could utilise the system to collaborate on version-agnostic local toolchains and to significantly reduce commit and check-out sizes.}
		\item {Implemented an memory-mapped AXI4 register interface wrapper around the Aurora 8B/10B by extending the respective Xilinx Baremetal drivers.}
	\end{cvitems}}
%---------------------------------------------------------

\cvposition
	{Senior Research Fellow} % Position
	{``Underwater Target Motion Analysis with Passive Sensors''} % Title
	{Supervisor: \href{http://www.tutorialpoint.org/ShovanBhaumik/index.html}{Dr Shovan Bhaumik}, Sponsor: Naval Physical \& Oceanographic Laboratory (DRDO)} % Subtitle
	{Control and Instrumentation Lab, \href{https://www.iitp.ac.in}{IIT Patna}, India} % Location
	{May 2018 -- Nov. 2018} % Date(s)
	{\begin{cvitems} % Description(s)
		\item {Implemented advanced filters for the Bearings-only Tracking problem - Distributed Extended Kalman Filter, Shifted Rayleigh Filter, Particle Filter with compound proposals, Particle Filter with MCMC, and Regularised Particle Filter.}
		\item {Simulated the performance of these filters on actual field manoeuvres provided by the sponsor, and prepared a comparative study as measured by time, resource and computational complexities, and target tracking accuracy. Concluded that Shifted Rayleigh Filter outperforms other filters in terms of computational complexity while still being superior at tracking the target.}
	\end{cvitems}}
%---------------------------------------------------------

\cvposition
	{B.Tech. Thesis\quad\textbar\quad Nominated for the Best B.Tech. Thesis award from Dept of Electrical Engineering} % Position
	{\href{https://github.com/haitmak/sigma}{``Hardware Architecture of a Family of Sigma-Point Kalman Filters for Bayesian Estimation''}} % Title
	{Supervisor: \href{http://www.tutorialpoint.org/ShovanBhaumik/index.html}{Dr Shovan Bhaumik}, Assistant Professor, IIT Patna} % Subtitle
	{\href{https://www.iitp.ac.in}{IIT Patna}, India} % Location
	{Aug. 2017 -- May 2018} % Date(s)
	{\begin{cvitems} % Description(s)
		\item {Designed and implemented a parallel architecture of Sigma-point Kalman filtering algorithms on an FPGA by independently conceptualised a parallel routine to the Cholesky decomposition of a positive-definite matrix in $\mathcal{O} \left( N \right)$ time complexity.}
		\item {Further optimised resource usage of parallel Cholesky decomposition architecture for maximum processor utilisation to achieve $\mathcal{O} \left( \frac{1}{4} N^2 \right)$ resource complexity, as compared to $\mathcal{O} \left( \frac{1}{2} N^2 \right)$ resource complexity of state-of-the-art.}
%		\item {Customised and implemented a matrix inversion routine to compute covariance inverses in $\mathcal{O} \left( N \right)$ time complexity.}
		\item {Implemented the parallel architectures using Verilog HDL and Xilinx Vivado on the Xilinx Zynq-7000 ZC702 and Digilent Nexys4 DDR FPGA boards, making use of open-source floating-point IPs and Xilinx Vivado IPs.}
		\item {Presented the final work product to the professors of the department and was one of the only two students to receive a unanimous 10 / 10 grade out of a cohort of 50 candidates.}
	\end{cvitems}}
%---------------------------------------------------------

\cvposition
	{Student Developer, \href{https://github.com/librecores/gsoc-museum-edsac}{``EDSAC Museum on FPGA''}\quad\textbar\quad \href{https://orconf.org/2017/\#edsac}{Presented at ORConf 2017 digital design conference}} % Title
	{\href{https://fossi-foundation.org}{Free and Open Source Silicon Foundation}} % Organization
	{Mentor: \href{http://www.jeremybennett.com}{Dr Jeremy Bennett}, \href{https://www.embecosm.com/about/meet-the-team/jeremy-bennett/}{Founder \& Chief Executive}, Embecosm Ltd} % Subtitle
	{\href{https://summerofcode.withgoogle.com/archive/2017/projects/6470218444439552/}{Google Summer of Code 2017}} % Location
	{May 2017 -- Aug. 2017} % Date(s)
	{\begin{cvitems} % Description(s)
		\item {Built a Verilog model of the historic EDSAC computer from original but incomplete documentation in collaboration with members of ``The EDSAC Replica Project'' team (TNMOC, Bletchley Park, UK).}
		\item {Programmed and simulated the EDSAC architecture and ISA on the \href{https://mystorm.uk/}{myStorm} Lattice iCE FPGA board using open-source toolchains, like \href{http://www.clifford.at/yosys/}{Yosys} and \href{http://iverilog.icarus.com}{iverilog}.}
%		\item {Designed and implemented a modified-UART communication protocol to support external I/O interfaces to the system.}
		\item {Coordinated with a team of younger students in the UK to build a hardware imitation of the EDSAC memory delay line, teleprinter and paper tape reader.}
		\item {Demonstrated the final work product at the \href{http://chiphack.org/chiphack-2017.html}{ChipHack 2017} workshop and \href{https://youtu.be/EZkJOyOcYiY}{presented at the ORConf 2017} digital design conference in Hebden Bridge, UK, for which full sponsorship was received.}
	\end{cvitems}}
%---------------------------------------------------------

\cvposition
	{Student Developer, \href{https://drive.google.com/drive/u/1/folders/0B-Cccp-WWmeuUlh5M3pxT0cyQm8}{``Read/Write Multiple Status Registers and Lock/Unlock Memory on SPI Chips''}} % Title
	{\href{https://www.coreboot.org}{Coreboot} (\href{https://www.flashrom.org/Flashrom}{Flashrom})} % Organization
	{} % Subtitle
	{\href{https://summerofcode.withgoogle.com/archive/2016/projects/5439533130711040/}{Google Summer of Code 2016}} % Location
	{Feb. 2016 -- Aug. 2016} % Date(s)
	{\begin{cvitems} % Description(s)
		\item {Designed a unified abstraction of multiple status registers in an SPI Flash-memory chip to provide a consistent interface between flashrom and diverse chip manufacturers.}
		\item {Programmed routines to lock/unlock memory space governed by status registers, handle configuration bits, access/lock OTP memory areas, and automatically generate memory protection maps for a selection of chips. Developed a CLI to expose these features, and finally tested the infrastructure using a Raspberry Pi and a Teensy development board.}
  		\item {Liaised with a Sales Executive of GigaDevie from China to arrange for engineering samples and add support for GigaDevice SPI chips.}
	\end{cvitems}}
%---------------------------------------------------------

% \cvposition
% 	{Summer Intern} % Job title
% 	{Aficionado Ventures} % Organization
% 	{} % Subtitle
% 	{New Delhi, India} % Location
% 	{May 2016 -- July 2016} % Date(s)
% 	{\begin{cvitems} % Description(s) of tasks/responsibilities
% 		\item {Aficionado was a startup helping restaurants procure quality produce from competitive collection of vendors.}
% 		\item {Designed reactive backend architecture - MeteorJS server, MongoDB database, Heroku/mLab web hosting, and Cordova for cross-platform mobile apps.}
% 		\item {Implemented MVP-stage marketplace platform from scratch with bilingual search for products in English/Hinglish.}
% 		\item {Worked with business team to conduct market research on restaurants and vendors.}
% 	\end{cvitems}}
% %---------------------------------------------------------

\end{cvexperience}
