\begin{cvexperience}

\cvExperienceTitle{EN}{M.\,Sc. Thesis ``Field-Programmable Gate Array based Real-Time Control and Simulation''} % Job title
\cvExperienceTitle{DE}{Masterarbeit ``Field-Programmable Gate Array basierte Echtzeitregelung und -simulation''} % Job title

\cvExperienceOrganisation{EN}{Institute of Energy and Climate Research (IEK-10), Forschungszentrum Juelich GmbH} % Organization
\cvExperienceOrganisation{DE}{Institut f{\"u}r Energie- und Klimaforschung (IEK-10) am Forschungszentrum J{\"u}lich GmbH} % Organization

\cvExperienceSubtitle{EN}{Supervisors: Univ.-Prof. Dr.-Ing. Andrea Benigni, Steffen Vogel, M.\,Sc., Dr.-Ing. Lukas Razik} % Subtitle
%% TODO @hatimak: German Subtitles
\cvExperienceSubtitle{DE}{Supervisors: Univ.-Prof. Dr.-Ing. Andrea Benigni, Steffen Vogel, M.\,Sc., Dr.-Ing. Lukas Razik} % Subtitle

\cvExperienceLocation{EN}{Aachen, Germany \ifbool{showEmojis} {\space\emoji{flag-germany}} {} } % Location
\cvExperienceLocation{DE}{Aachen, Deutschland \ifbool{showEmojis} {\space\emoji{flag-germany}} {} } % Location

\cvExperienceDates{EN}{Apr. 2021 -- Dec. 2021} % Date(s)
\cvExperienceDates{DE}{Apr. 2021 -- Dez. 2021} % Date(s)

\cvExperienceDescription{EN}{%
	\begin{cvitems}
		\item {Developed design with soft-core microprocessors to rapidly prototype control-loop algorithms for FPGA-based real-time simulators.}
		\item {Introduced control and data-logger soft-cores dedicated to running control algorithms at switching frequency and logging simulation data.}
		\item {Conceptualised heterogenous architecture of control and data-logger soft-cores dedicated to running control algorithms at switching frequency and logging simulation data.}
%		\item {Conceptualised heterogeneous architecture of multiple dedicated soft-core microprocessors, enabling hierarchical control-loop system designs.}
		\item {Implemented digital design based on soft-core MicroBlaze microprocessor from Xilinx on Virtex Ultrascale+ VCU118 board.}
	\end{cvitems}%
}
\cvExperienceDescription{DE}{%
	\begin{cvitems}
		\item {Entwickelte ein Design mit Soft-Core-Mikroprozessoren, um Regelkreisalgorithmen f{\"u}r FPGA-basierte Echtzeitsimulatoren schnell zu prototypisieren.}
		\item {Entwarf von Control- und Data-Logger-Softcores f{\"u}r die Ausf{\"u}hrung von Steuerungsalgorithmen mit Schaltfrequenz und die Speicherung von Simulationsdaten.}
		\item {Entwarf einer heterogenen Architektur von Control- und Data-Logger-Softcores für die Ausführung von Steuerungsalgorithmen mit Schaltfrequenz und die Aufzeichnung von Simulationsdaten.}
%		\item {Konzipierte heterogene Architektur mehrerer dedizierter Soft-Core-Mikroprozessoren, die hierarchische Regelkreissysteme erm{\"o}glicht.}
 		\item {Implementierte ein digitales Design basiert auf einem Soft-Core MicroBlaze Mikroprozessor von Xilinx auf einem Virtex Ultrascale+ VCU118 Board.}
	\end{cvitems}%
}

%\cvExperienceFlush
%---------------------------------------------------------

\cvExperienceTitle{EN}{Intern} % Job title
\cvExperienceTitle{DE}{Praktikant} % Job title

\cvExperienceOrganisation{EN}{Institute of Energy and Climate Research (IEK-10), Forschungszentrum Juelich GmbH} % Organization
\cvExperienceOrganisation{DE}{Institut f{\"u}r Energie- und Klimaforschung (IEK-10) am Forschungszentrum J{\"u}lich GmbH} % Organization

\cvExperienceSubtitle{EN}{Supervisor: Dr.-Ing. Lukas Razik, Head, HPC Department, IEK-10} % Subtitle
\cvExperienceSubtitle{DE}{Supervisor: Dr.-Ing. Lukas Razik, Head, HPC Department, IEK-10} % Subtitle

\cvExperienceLocation{EN}{Remote \ifbool{showEmojis} {\space\emoji{globe-showing-europe-africa}} {} } % Location
\cvExperienceLocation{DE}{Home Office \ifbool{showEmojis} {\space\emoji{globe-showing-europe-africa}} {} } % Location

\cvExperienceDates{EN}{Oct. 2020 -- Feb. 2021} % Date(s)
\cvExperienceDates{DE}{Okt. 2020 -- Feb. 2021} % Date(s)

\cvExperienceDescription{EN}{%
	\begin{cvitems}
		\item {Implemented power sytems models using High-Level Synthesis designs for RTL co-simulation and real-time simulation on FPGA.}
		\item {Extended HLS models with memory-mapped \texttt{AXI4} register interface. Verified hardware models on Virtex-7 VC707 FPGA board using remote debugging.}
  		\item {Developed Makefile pipeline on Linux for \texttt{ORTiS} code generation, Vivado High-Level Synthesis, IP Integrator and FPGA bitstream generation stages.}
	\end{cvitems}%
}
\cvExperienceDescription{DE}{%
	\begin{cvitems}
		\item {Implementierte Stromnetzmodelle mit High-Level Synthesis-Designs f{\"u}r RTL Co-Simulation und Echtzeitsimulation auf FPGA.}
		\item {Erweiterte HLS-Modelle mit memory-mapped \texttt{AXI4}-Register-Schnittstellen. Verifizierte Hardware-Modelle auf Virtex-7 VC707 FPGA-Board mit Remote-Debugging.}
  		\item {Entwickelte Makefile-Pipeline unter Linux f{\"u}r \texttt{ORTiS}-Code-Generierung, Vivado High-Level Synthesis, IP Integrator und FPGA-Bitstream-Generierung.}
	\end{cvitems}%
}

\cvExperienceFlush
%---------------------------------------------------------

\cvExperienceTitle{EN}{Research Assistant, Development of FPGA Systems} % Job title
\cvExperienceTitle{DE}{Wissenschaftliche Hilfskraft (Entwicklung von FPGA-Systeme)} % Job title

\cvExperienceOrganisation{EN}{Institute for Automation of Complex Power Systems, E.ON Energy Research Centre} % Organization
\cvExperienceOrganisation{DE}{Institute for Automation of Complex Power Systems, E.ON Energy Research Centre} % Organization

\cvExperienceSubtitle{EN}{Supervisor: \href{https://www.acs.eonerc.rwth-aachen.de/go/id/msul}{Steffen Vogel, M.\, Sc.}} % Subtitle
\cvExperienceSubtitle{DE}{Supervisor: \href{https://www.acs.eonerc.rwth-aachen.de/go/id/msul}{Steffen Vogel, M.\, Sc.}} % Subtitle

\cvExperienceLocation{EN}{Aachen, Germany \ifbool{showEmojis} {\space\emoji{flag-germany}} {} } % Location
\cvExperienceLocation{DE}{Aachen, Deutschland \ifbool{showEmojis} {\space\emoji{flag-germany}} {} } % Location

\cvExperienceDates{EN}{May 2019 -- Sept. 2020} % Date(s)
\cvExperienceDates{DE}{Mai 2019 -- Sept. 2020} % Date(s)

\cvExperienceDescription{EN}{%
	\begin{cvitems}
		\item {Integrated Xilinx FPGA boards into \href{https://villas.fein-aachen.org}{VILLAS} co-simulation platform by designing an architecture built on top of Aurora 8B/10B serial protocol.}
		\item {Engineered Tcl-Makefile system of scripts to automate design generation and bitstream compilation leading to version-agnostic local toolchain use.}
%		\item {Engineered Tcl-Makefile system of scripts to automate design generation and bitstream compilation.}
		\item {Developed FPGA designs in Verilog/VHDL and bare-metal driver programs in C/C++ for FPGA firmware.}
	\end{cvitems}%
}
\cvExperienceDescription{DE}{%
	\begin{cvitems}
		\item {Integrierte Xilinx-FPGA-Boards in die VILLAS-Cosimulations-Plattform durch den Aufbau einer Architektur auf dem seriellen Aurora 8B/10B-Protokoll.}
		\item {Entwickelte ein Tcl-Makefile-System mit Skripten zur Automatisierung der Design-Generierung und Bitstream-Kompilierung, das die versionsunabh{\"a}ngige Verwendung lokaler Toolchains erm{\"o}glicht.}
%		\item {Entwickelte ein Tcl-Makefile-System mit Skripten zur Automatisierung der Design-Generierung und Bitstream-Kompilierung.}
		\item {Entwickelte FPGA-Designs in Verilog/VHDL und Bare-Metal-Driver-Programme in C/C++ f{\"u}r FPGA-Firmware.}
	\end{cvitems}%
}

\cvExperienceFlush
%---------------------------------------------------------

\cvExperienceTitle{EN}{Senior Research Engineer} % Position
\cvExperienceTitle{DE}{Senior Research Engineer} % Position

\cvExperienceOrganisation{EN}{``Underwater Target Motion Analysis with Passive Sensors'',\newline\href{https://www.drdo.gov.in/labs-and-establishments/naval-physical-oceanographic-laboratory-npol}{Naval Physical \& Oceanographic Laboratory} (DRDO), Ministry of Defence, Govt. of India} % Title
\cvExperienceOrganisation{DE}{``Underwater Target Motion Analysis with Passive Sensors'',\newline\href{https://www.drdo.gov.in/labs-and-establishments/naval-physical-oceanographic-laboratory-npol}{Naval Physical \& Oceanographic Laboratory} (DRDO), Ministry of Defence, Govt. of India} % Title

\cvExperienceSubtitle{EN}{Supervisor: \href{http://www.tutorialpoint.org/ShovanBhaumik/index.html}{Dr Shovan Bhaumik}, Sponsor: \href{https://www.drdo.gov.in/labs-and-establishments/naval-physical-oceanographic-laboratory-npol}{Naval Physical \& Oceanographic Laboratory (DRDO)}} % Subtitle
\cvExperienceSubtitle{DE}{Supervisor: \href{http://www.tutorialpoint.org/ShovanBhaumik/index.html}{Dr Shovan Bhaumik}, Sponsor: \href{https://www.drdo.gov.in/labs-and-establishments/naval-physical-oceanographic-laboratory-npol}{Naval Physical \& Oceanographic Laboratory (DRDO)}} % Subtitle

\cvExperienceLocation{EN}{\href{https://www.iitp.ac.in}{IIT Patna}, India \ifbool{showEmojis} {\space\emoji{flag-india}} {} } % Location
\cvExperienceLocation{DE}{\href{https://www.iitp.ac.in}{IIT Patna}, Indien \ifbool{showEmojis} {\space\emoji{flag-india}} {} } % Location

\cvExperienceDates{EN}{May 2018 -- Nov. 2018} % Date(s)
\cvExperienceDates{DE}{Mai 2018 -- Nov. 2018} % Date(s)

\cvExperienceDescription{EN}{%
	\begin{cvitems} % Description(s)
		\item {Implemented advanced tracking filters in MATLAB, like Distributed Extended Kalman Filter, Shifted Rayleigh FilterSequential Monte Carlo methods, for the Bearings-only Tracking problem.}
		\item {Implemented advanced tracking filters in MATLAB for the Bearings-only Tracking problem.}
		\item {Simulated performance of modern filters on real field manoeuvre data from Indian Navy, and prepared comparative study.}
%		\item {Concluded Shifted Rayleigh Filter outperforms other filters in terms of computational complexity and tracking accuracy.}
	\end{cvitems}%
}
\cvExperienceDescription{DE}{%
	\begin{cvitems} % Description(s)
		\item {Implementierte fortgeschrittene Tracking-Filters in MATLAB, wie Distributed Extended Kalman Filter, Shifted Rayleigh Filter und Sequenzielle Monte-Carlo-Methode, f{\"u}r das Bearings-only Tracking Problem.}
		\item {Implementierte fortgeschrittene Tracking-Filters in MATLAB f{\"u}r das Bearings-only Tracking-Problem.}
		\item {Simulierte die Leistung von modernen Filtern anhand realer Manöverdaten der indischen Marine und erstellte einen komparativen Studienbericht.}
%		\item {Der Shifted-Rayleigh-Filter übertrifft andere Filtersysteme in Bezug auf den Komplexitätsund die Tracking-Genauigkeit.}
	\end{cvitems}%
}

\cvExperienceFlush
%---------------------------------------------------------

\cvExperienceTitle{EN}{B.\,Tech. Thesis \href{https://github.com/hatimak/sigma}{``Hardware Architecture of a Family of Sigma-Point Kalman Filters for Bayesian Estimation''}} % Position
\cvExperienceTitle{DE}{Bachelorarbeit \href{https://github.com/hatimak/sigma}{``Hardware Architecture of a Family of Sigma-Point Kalman Filters for Bayesian Estimation''}} % Position

\cvExperienceOrganisation{EN}{Control and Instrumentation Lab} % Title
\cvExperienceOrganisation{DE}{Control and Instrumentation Lab} % Title

\cvExperienceSubtitle{EN}{Supervisor: \href{http://www.tutorialpoint.org/ShovanBhaumik/index.html}{Dr Shovan Bhaumik}, Assistant Professor, IIT Patna} % Subtitle
\cvExperienceSubtitle{DE}{Supervisor: \href{http://www.tutorialpoint.org/ShovanBhaumik/index.html}{Dr Shovan Bhaumik}, Assistant Professor, IIT Patna} % Subtitle

\cvExperienceLocation{EN}{\href{https://www.iitp.ac.in}{IIT Patna}, India \ifbool{showEmojis} {\space\emoji{flag-india}} {} } % Location
\cvExperienceLocation{DE}{\href{https://www.iitp.ac.in}{IIT Patna}, Indien \ifbool{showEmojis} {\space\emoji{flag-india}} {} } % Location

\cvExperienceDates{EN}{Aug. 2017 -- May 2018} % Date(s)
\cvExperienceDates{DE}{Aug. 2017 -- Mai 2018} % Date(s)

\cvExperienceDescription{EN}{%
	\begin{cvitems} % Description(s)
		\item {Designed and implemented a parallel architecture of Sigma-point Kalman filtering algorithms on FPGA.}
		\item {Conceptualised novel parallel routine for Cholesky matrix decomposition; improvement from $\mathcal{O} \left( N^3 \right)$ to $\mathcal{O} \left( N \right)$ time complexity.}
		\item {Optimised resource usage of Cholesky decomposition architecture for double utilisation at same processor count.}
%		\item {Customised and implemented a matrix inversion routine to compute covariance inverses in $\mathcal{O} \left( N \right)$ time complexity.} % TODO @hatimak: German version not written yet
		\item {Implemented parallel designs in Verilog HDL using Vivado and open-source IPs on Zynq-7000 ZC702 and Digilent Nexys4 DDR FPGA boards.}
%		\item {One of only two students to receive 10 / 10 grade from cohort of 50 candidates. Nominated for Best B.\,Tech. Thesis award from Dept of Electrical Engineering.}
	\end{cvitems}%
}
\cvExperienceDescription{DE}{%
	\begin{cvitems} % Description(s)
		\item {Entwarf und implementierte eine parallele Architektur von Sigma-Point Kalman-Filteralgorithmen auf FPGA.}
		\item {Konzipierte eine neue parallele Algorithm f{\"u}r die Dekomposition der Cholesky-Matrix; Verbesserung der Zeitkomplexit{\"a}t von $\mathcal{O} \left( N^3 \right)$ zu $\mathcal{O} \left( N \right)$.}
		\item {Optimierte Ressourcennutzung der Cholesky-Dekompositionsarchitektur f{\"u}r doppelte Nutzung bei gleicher Prozessoranzahl.}
		\item {Implementierte parallele Designs in Verilog HDL unter Verwendung von Vivado und Open-Source-IPs auf Zynq-7000 ZC702 und Digilent Nexys4 DDR FPGA-Boards.}
%		\item {Einer von nur zwei Studenten, die aus einem Jahrgang von 50 Kandidaten die höchste Note 10 / 10 erhielten. Nominiert für den Preis für die beste B.Tech. Thesis Award von der Fakultät der Elektrotechnik.}
	\end{cvitems}%
}

%\cvExperienceFlush
%---------------------------------------------------------

\cvExperienceTitle{EN}{Software Developer} % Title
\cvExperienceTitle{DE}{Softwareentwickler} % Title

\cvExperienceOrganisation{EN}{\href{https://fossi-foundation.org}{Free and Open Source Silicon Foundation}, \href{https://github.com/librecores/gsoc-museum-edsac}{``EDSAC Museum on FPGA''}} % Organization
\cvExperienceOrganisation{DE}{\href{https://fossi-foundation.org}{Free and Open Source Silicon Foundation}, \href{https://github.com/librecores/gsoc-museum-edsac}{``EDSAC Museum on FPGA''}} % Organization

\cvExperienceSubtitle{EN}{Mentor: \href{http://www.jeremybennett.com}{Dr Jeremy Bennett}, \href{https://www.embecosm.com/about/meet-the-team/jeremy-bennett/}{Founder \& Chief Executive}, Embecosm Ltd} % Subtitle
\cvExperienceSubtitle{DE}{Mentor: \href{http://www.jeremybennett.com}{Dr Jeremy Bennett}, \href{https://www.embecosm.com/about/meet-the-team/jeremy-bennett/}{Founder \& Chief Executive}, Embecosm Ltd} % Subtitle

\cvExperienceLocation{EN}{\href{https://summerofcode.withgoogle.com/archive/2017/projects/6470218444439552/}{Google Summer of Code 2017} \ifbool{showEmojis} {\space\emoji{globe-showing-asia-australia}} {} } % Location
\cvExperienceLocation{DE}{\href{https://summerofcode.withgoogle.com/archive/2017/projects/6470218444439552/}{Google Summer of Code 2017} \ifbool{showEmojis} {\space\emoji{globe-showing-asia-australia}} {} } % Location

\cvExperienceDates{EN}{May 2017 -- Aug. 2017} % Date(s)
\cvExperienceDates{DE}{Mai 2017 -- Aug. 2017} % Date(s)

\cvExperienceDescription{EN}{%
	\begin{cvitems} % Description(s)
		\item {Built Verilog model of historic EDSAC computer from original but incomplete documentation in collaboration with experts from The National Museum of Computing, UK.}
		\item {Programmed and simulated EDSAC architecture and ISA on \href{https://mystorm.uk/}{myStorm} Lattice iCE FPGA board using open-source toolchains, like \href{http://www.clifford.at/yosys/}{\texttt{Yosys}} and \href{http://iverilog.icarus.com}{\texttt{iverilog}}.}
		\item {Designed and implemented modified-UART communication protocol to support external embedded I/O interfaces for HiL application.}
		\item {Coordinated with team of students in UK to build hardware imitation of EDSAC memory delay line, teleprinter and paper tape reader.}
		\item {Demonstrated final work product at \href{http://chiphack.org/chiphack-2017.html}{ChipHack 2017 workshop} and \href{https://youtu.be/EZkJOyOcYiY}{presented} at \href{https://orconf.org/2017/\#edsac}{ORConf 2017 digital design conference} in Hebden Bridge, UK.}
	\end{cvitems}%
}
\cvExperienceDescription{DE}{%
	\begin{cvitems} % Description(s)
		\item {Baute eines Verilog-Modells eines historischen EDSAC-Computers auf der ursprünglichen aber unvollständigen Dokumentation in Zusammenarbeit mit Experten des National Museum of Computing, UK.}
		\item {Programmierte und simulierte die EDSAC-Architektur und ISA auf dem \href{https://mystorm.uk/}{myStorm} Lattice iCE FPGA-Board unter Verwendung von Open-Source-Toolchains wie \href{http://www.clifford.at/yosys/}{\texttt{Yosys}} und \href{http://iverilog.icarus.com}{\texttt{iverilog}}.}
		\item {Entwarf und implementierte ein modifiziertes UART-Kommunikationsprotokoll zur Unterst{\"u}tzung externer Embedded-I/O-Schnittstellen f{\"u}r HiL-Anwendungen.}
		\item {Koordinierte mit einem Team von Studenten im UK den Bau einer Hardware-Imitation der EDSAC-Memory-Delay-Line, eines Teleprinter und eines Paper-Tape-Reader.}
		\item {\href{https://youtu.be/EZkJOyOcYiY}{Pr{\"a}sentierte} das Arbeitsergebnis auf dem \href{http://chiphack.org/chiphack-2017.html}{ChipHack-Workshop 2017} und stellte es auf der \href{https://orconf.org/2017/\#edsac}{ORConf-Digital-Design-Konferenz 2017} in UK vor.}
	\end{cvitems}%
}

\cvExperienceFlush
%---------------------------------------------------------

\cvExperienceTitle{EN}{Software Developer} % Title
\cvExperienceTitle{DE}{Softwareentwickler} % Title

\cvExperienceOrganisation{EN}{\href{https://www.coreboot.org}{Coreboot} (\href{https://www.flashrom.org/Flashrom}{Flashrom}), \href{https://drive.google.com/drive/u/1/folders/0B-Cccp-WWmeuUlh5M3pxT0cyQm8}{``Read/Write Multiple Status Registers and Lock/Unlock Memory on SPI Chips''}} % Organization
\cvExperienceOrganisation{DE}{\href{https://www.coreboot.org}{Coreboot} (\href{https://www.flashrom.org/Flashrom}{Flashrom}), \href{https://drive.google.com/drive/u/1/folders/0B-Cccp-WWmeuUlh5M3pxT0cyQm8}{``Read/Write Multiple Status Registers and Lock/Unlock Memory on SPI Chips''}} % Organization

\cvExperienceSubtitle{EN}{} % Subtitle
\cvExperienceSubtitle{DE}{} % Subtitle

\cvExperienceLocation{EN}{\href{https://summerofcode.withgoogle.com/archive/2016/projects/5439533130711040/}{Google Summer of Code 2016} \ifbool{showEmojis} {\space\emoji{globe-showing-asia-australia}} {} } % Location
\cvExperienceLocation{DE}{\href{https://summerofcode.withgoogle.com/archive/2016/projects/5439533130711040/}{Google Summer of Code 2016} \ifbool{showEmojis} {\space\emoji{globe-showing-asia-australia}} {} } % Location

\cvExperienceDates{EN}{Apr. 2016 -- Aug. 2016} % Date(s)
\cvExperienceDates{DE}{Apr. 2016 -- Aug. 2016} % Date(s)

\cvExperienceDescription{EN}{%
	\begin{cvitems} % Description(s)
		\item {Designed unified abstraction of multiple status registers in SPI Flash-memory chips across diverse chip manufacturers.}
		\item {Programmed routines to lock/unlock memory areas, handle configuration bits, and automatically generate memory protection maps.}
		\item {Developed CLI to expose new features, and tested infrastructure using Raspberry Pi and Teensy development board.}
		\item {Liaised with Sales Executive of GigaDevice from China to arrange for engineering samples and add support for those chips.}
	\end{cvitems}%
}
\cvExperienceDescription{DE}{%
	\begin{cvitems} % Description(s)
		\item {Entwickelte eine einheitliche Abstraktion von Statusregistern in SPI Flash-Speicherchips von verschiedenen Chip-Herstellern.}
		\item {Programmierte Functions zum Sperren/Entsperren von Speicherpl{\"a}tzen, zum Umgang mit Konfigurationsbits und zur automatischen Generierung von Speicherschutzmaps.}
		\item {Entwickelte CLI, um neue Funktionen bereitzustellen, und testete die Infrastruktur mit Raspberry Pi und Teensy-Development-Board.}
		\item {Kontaktaufnahme mit dem Verkaufsleiter von GigaDevice aus China, um Engineering-Probe-Chips zu besorgen und Unterst{\"u}tzung f{\"u}r diese Chips einzuf{\"u}hren.}
	\end{cvitems}%
}

\cvExperienceFlush
%---------------------------------------------------------

% \cvposition
% 	{Summer Intern} % Job title
% 	{Aficionado Ventures} % Organization
% 	{} % Subtitle
% 	{New Delhi, India} % Location
% 	{May 2016 -- July 2016} % Date(s)
% 	{\begin{cvitems} % Description(s) of tasks/responsibilities
% 		\item {Aficionado was a startup helping restaurants procure quality produce from competitive collection of vendors.}
% 		\item {Designed reactive backend architecture - MeteorJS server, MongoDB database, Heroku/mLab web hosting, and Cordova for cross-platform mobile apps.}
% 		\item {Implemented MVP-stage marketplace platform from scratch with bilingual search for products in English/Hinglish.}
% 		\item {Worked with business team to conduct market research on restaurants and vendors.}
% 	\end{cvitems}}
% %---------------------------------------------------------

\end{cvexperience}
