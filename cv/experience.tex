\cvsection{Experience}

\begin{cvexperience}

\cvposition
	{Senior Research Fellow} % Position
	{``Underwater Target Motion Analysis with Passive Sensors''} % Title
	{Supervisor: \href{http://www.tutorialpoint.org/ShovanBhaumik/index.html}{Dr Shovan Bhaumik}, Sponsor: Naval Physical \& Oceanographic Laboratory (DRDO)} % Subtitle
	{Control and Instrumentation Lab, IIT Patna} % Location
	{May 2018 -- PRESENT} % Date(s)
	{\begin{cvitems} % Description(s)
		\item {Developing hardware implementation of Bearings-Only Tracking problem using Shifted Rayleigh Filter.}
		\item {Devising novel architectures for passive sensor networks to be used with Kalman-Consensus Filtering paradigm, particularly in context of underwater target tracking.}
		\item {Prototyping decentralised Distributed Kalman Filtering algorithm on FPGA with final objective to fabricate IC that alleviates performance, space and thermal complexity at computational nodes of estimator network.}
	\end{cvitems}}
%---------------------------------------------------------

\cvposition
	{B.Tech. Thesis\quad\textbar\quad Nominated for the Best B.Tech. Thesis award from Dept of Electrical Engineering} % Position
	{\href{https://github.com/haitmak/sigma}{``Hardware Architecture of a Family of Sigma-Point Kalman Filters for Bayesian Estimation''}} % Title
	{Supervisor: \href{http://www.tutorialpoint.org/ShovanBhaumik/index.html}{Dr Shovan Bhaumik}} % Subtitle
	{IIT Patna} % Location
	{Aug. 2017 -- May 2018} % Date(s)
	{\begin{cvitems} % Description(s)
		\item {Developed parallel architecture of Sigma-point filtering algorithms like UKF, CKF, etc.}
		\item {Designed and implemented on FPGA a parallel architecture to compute the Cholesky decomposition of a positive-definite matrix in $\mathcal{O} \left( N \right)$ time complexity.}
		\item {Further optimised resource usage of parallel Cholesky decomposition architecture for maximum processor utilisation to achieve $\mathcal{O} \left( \frac{1}{4} N^2 \right)$ resource complexity, as compared to $\mathcal{O} \left( \frac{1}{2} N^2 \right)$ resource complexity of state-of-the-art.}
		\item {Using parallel Cholesky architecture, formulated and implemented linear time complexity matrix inverse routine to compute covariance inverses required by sigma-point filtering algorithms.}
		\item {Implemented Gaussian RNG based on CDF Inversion method using FPGA amenable LUT-SR Uniform RNG.}
		\item {Implemented all the parallel architectures using Verilog HDL and Xilinx Vivado on Xilinx Zynq-7000 ZC702 and Digilent Nexys4 DDR FPGA boards. Made use of open-source floating-point IP and several Xilinx Vivado IP.}
	\end{cvitems}}
%---------------------------------------------------------

\cvposition
	{Student Developer, \href{https://github.com/librecores/gsoc-museum-edsac}{``EDSAC Museum on FPGA''}} % Title
	{\href{https://fossi-foundation.org}{Free and Open Source Silicon Foundation}} % Organization
	{\href{https://orconf.org/2017/\#edsac}{Presented at ORConf 2017, Hebden Bridge, UK}} % Subtitle
	{\href{https://summerofcode.withgoogle.com/archive/2017/projects/6470218444439552/}{Google Summer of Code 2017}} % Location
	{May 2017 -- Aug. 2017} % Date(s)
	{\begin{cvitems} % Description(s)
		\item {Built a model of EDSAC from original albeit incomplete documentation, and from correspondences with members of ``The EDSAC Replica Project'' team (TNMOC, Bletchley Park, UK).}
		\item {Replicated EDSAC architecture and ISA on \href{https://mystorm.uk/}{myStorm} Lattice iCE FPGA board using Verilog HDL and open-source toolchain \href{http://www.clifford.at/yosys/}{Yosys}.}
		\item {Designed and implemented modified-UART communication protocol for external extensible I/O interfaces.}
		\item {Helped a team of students in UK build hardware reproduction of EDSAC memory delay line, teleprinter and paper tape reader.}
		\item {Final work product used at \href{http://chiphack.org/chiphack-2017.html}{ChipHack 2017} workshop and \href{https://youtu.be/EZkJOyOcYiY}{presented work at ORConf 2017} conference during Wuthering Bytes festival in Hebden Bidge, UK.}
	\end{cvitems}}
%---------------------------------------------------------

\cvposition
	{Student Developer, \href{https://drive.google.com/drive/u/1/folders/0B-Cccp-WWmeuUlh5M3pxT0cyQm8}{``Read/Write Multiple Status Registers and Lock/Unlock Memory on SPI Chips''}} % Title
	{\href{https://www.coreboot.org}{Coreboot} (\href{https://www.flashrom.org/Flashrom}{Flashrom})} % Organization
	{} % Subtitle
	{\href{https://summerofcode.withgoogle.com/archive/2016/projects/5439533130711040/}{Google Summer of Code 2016}} % Location
	{Feb. 2016 -- Aug. 2016} % Date(s)
	{\begin{cvitems} % Description(s)
		\item {Designed multiple status registers model to abstract chip diversities across manufacturers into single consistent interface.}
		\item {Developed routines to lock/unlock memory space governed by bits in status registers, handle configuration bits, and automatically generate memory protection maps for some chips.}
		\item {Added functionality to access/lock OTP memory areas.}
		\item {Developed CLI to expose new infrastructure. Tested on physical GigaDevice SPI chips using Raspberry Pi and Teensy.}
	\end{cvitems}}
%---------------------------------------------------------

\cvposition
	{Summer Intern} % Job title
	{Aficionado Ventures} % Organization
	{} % Subtitle
	{New Delhi, India} % Location
	{May 2016 -- July 2016} % Date(s)
	{\begin{cvitems} % Description(s) of tasks/responsibilities
		\item {Aficionado was a startup helping restaurants procure quality produce from competitive collection of vendors.}
		\item {Designed reactive backend architecture - MeteorJS server, MongoDB database, Heroku/mLab web hosting, and Cordova for cross-platform mobile apps.}
		\item {Implemented MVP-stage marketplace platform from scratch with bilingual search for products in English/Hinglish.}
		\item {Worked with business team to conduct market research on restaurants and vendors.}
	\end{cvitems}}
%---------------------------------------------------------

\end{cvexperience}
