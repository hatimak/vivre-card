\cvsection{Research}

\begin{cventries}

\cventry
	{Senior Research Fellow\quad\textbar\quad Supervisor: Dr Shovan Bhaumik} % Position
	{Underwater Target Motion Analysis with Passive Sensors} % Project title
	{Control and Instrumentation Lab, IIT Patna} % Location
	{May 2018 -- PRESENT} % Date(s)
	{\begin{cvitems} % Description(s) of tasks/responsibilities
		\item {Developing hardware implementation of Bearings-Only Tracking problem using Shifted Rayleigh Filter.}
		\item {Devising novel architectures for passive sensor networks to be used with Kalman-Consensus Filtering paradigm, particularly in context of underwater target tracking.}
		\item {Prototyping decentralised Distributed Kalman Filtering algorithm on FPGA with final objective to fabricate an IC that alleviates performance, space and thermal complexity at computational nodes of estimator network.}
	\end{cvitems}}
%---------------------------------------------------------

\cventry
	{B.Tech. Thesis\quad\textbar\quad Supervisor: Dr Shovan Bhaumik} % Position
	{Hardware Architecture of a Family of Sigma-Point Kalman Filters for Bayesian Estimation} % Project title
	{IIT Patna} % Location
	{Aug. 2017 -- May 2018} % Date(s)
	{\begin{cvitems} % Description(s) of tasks/responsibilities
		\item {Developed parallel architecture of Sigma-point filtering algorithms like UKF, CKF, etc.}
		\item {Designed and implemented on FPGA a parallel architecture to compute the Cholesky decomposition of a positive-definite matrix in $\mathcal{O} \left( N \right)$ time complexity.}
		\item {Further optimised resource usage of parallel Cholesky decomposition architecture for maximum processor utilisation to achieve $\mathcal{O} \left( \frac{1}{4} N^2 \right)$ resource complexity, as compared to $\mathcal{O} \left( \frac{1}{2} N^2 \right)$ resource complexity of state-of-the-art.}
		\item {Using parallel Cholesky architecture, formulated and implemented linear time complexity matrix inverse routine to compute covariance inverses required by sigma-point filtering algorithms.}
		\item {Implemented Gaussian RNG based on CDF Inversion method using FPGA amenable LUT-SR Uniform RNG.}
		\item {Implemented all the parallel architectures using Verilog HDL and Xilinx Vivado on Xilinx Zynq-7000 ZC702 and Digilent Nexys4 DDR FPGA boards. Made use of open-source floating-point IP and several Xilinx Vivado IP.}
	\end{cvitems}}
%---------------------------------------------------------

\end{cventries}
