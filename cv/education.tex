\begin{cveducation}

\cvEducationDegree{EN}{M.\,Sc.} % Degree
\cvEducationDegree{DE}{M.\,Sc.} % Degree

\cvEducationDegreeName{EN}{Electrical Engineering, Information Technology and Computer Engineering} % Degree name
\cvEducationDegreeName{DE}{Elektrotechnik, Informationstechnik und Technische Informatik} % Degree name

\cvEducationInstitution{EN}{RWTH Aachen University} % Institution
\cvEducationInstitution{DE}{RWTH Aachen Universit{\"a}t} % Institution

\cvEducationLocation{EN}{Aachen, Germany \ifbool{showEmojis} {\space\emoji{flag-germany}} {} } % Location
\cvEducationLocation{DE}{Aachen, Deutschland \ifbool{showEmojis} {\space\emoji{flag-germany}} {} } % Location

\cvEducationDates{EN}{Apr. 2019 -- Oct. 2022} % Date(s)
\cvEducationDates{DE}{Apr. 2019 -- Okt. 2022} % Date(s)

\cvEducationGrade{EN}{2,1 (German) = 8 / 10 (Indian)} % Score
\cvEducationGrade{DE}{2,1} % Score

% Comments, like Thesis, or relevant coursework
\cvEducationDescription{EN}{%
Thesis: ``Field-Programmable Gate Array based Real-Time Control and Simulation''
\begin{cvInnerItems}%
	\item {Developed design with soft-core microprocessors to rapidly prototype control-loop algorithms for FPGA-based real-time simulators.}%
	\item {Conceptualised heterogenous architecture of control and data-logger soft-cores dedicated to running control algorithms at switching frequency and logging simulation data.}%
	\item {Implemented digital design based on soft-core MicroBlaze microprocessor from Xilinx on Virtex Ultrascale+ VCU118 board.}%
\end{cvInnerItems}%
}
\cvEducationDescription{DE}{%
Masterarbeit: ``Field-Programmable Gate Array basierte Echtzeitregelung und -simulation''
\begin{cvInnerItems}%
	\item {Entwickelte ein Design mit Soft-Core-Mikroprozessoren, um Regelkreisalgorithmen f{\"u}r FPGA-basierte Echtzeitsimulatoren schnell zu prototypisieren.}%
	\item {Entwarf einer heterogenen Architektur von Control- und Data-Logger-Softcores für die Ausführung von Steuerungsalgorithmen mit Schaltfrequenz und die Aufzeichnung von Simulationsdaten.}%
	\item {Implementierte ein digitales Design basiert auf einem Soft-Core MicroBlaze Mikroprozessor von Xilinx auf einem Virtex Ultrascale+ VCU118 Board.}%
\end{cvInnerItems}%
}

\cvSchoolFlush
%---------------------------------------------------------

\cvEducationDegree{EN}{B.\,Tech.} % Degree
\cvEducationDegree{DE}{B.\,Tech.} % Degree

\cvEducationDegreeName{EN}{Electrical Engineering} % Degree name
\cvEducationDegreeName{DE}{Electrical Engineering} % Degree name

\cvEducationInstitution{EN}{Indian Institute of Technology Patna} % Institution
\cvEducationInstitution{DE}{Indian Institute of Technology Patna} % Institution

\cvEducationLocation{EN}{Bihta (Patna), India \ifbool{showEmojis} {\space\emoji{flag-india}} {} } % Location
\cvEducationLocation{DE}{Bihta (Patna), Indien \ifbool{showEmojis} {\space\emoji{flag-india}} {} } % Location

\cvEducationDates{EN}{July 2014 -- May 2018} % Date(s)
\cvEducationDates{DE}{Juli 2014 -- Mai 2018} % Date(s)

\cvEducationGrade{EN}{7.32 / 10} % Score
\cvEducationGrade{DE}{7.32 / 10 (indische) = 2,3 (deutsche)} % Score

% Comments, like Thesis, or relevant coursework
\cvEducationDescription{EN}{%
Thesis: ``Hardware Architecture of a Family of Sigma-Point Kalman Filters for Bayesian Estimation''
\begin{cvInnerItems} % Description(s)
	\item {Designed and implemented a parallel architecture of Sigma-point Kalman filtering algorithms on FPGA resulting in improvement from $\mathcal{O} \left( N^3 \right)$ to $\mathcal{O} \left( N \right)$ time complexity.}%
	\item {Conceptualised novel parallel routine for Cholesky matrix decomposition; improvement from $\mathcal{O} \left( N^3 \right)$ to $\mathcal{O} \left( N \right)$ time complexity.}%
	\item {Implemented parallel designs in Verilog HDL using Vivado and open-source IPs on Zynq-7000 ZC702 and Digilent Nexys4 DDR FPGA boards.}%
\end{cvInnerItems}%
}
\cvEducationDescription{DE}{%
Bachelorarbeit: ``Hardware Architecture of a Family of Sigma-Point Kalman Filters for Bayesian Estimation''
\begin{cvInnerItems} % Description(s)
	\item {Entwarf und implementierte eine parallele Architektur von Sigma-Point-Kalman-Filteralgorithmen auf FPGA, die zu einer Verbesserung von $\mathcal{O} \left( N^3 \right)$ zu $\mathcal{O} \left( N \right)$ Zeitkomplexit{\"a}t f{\"u}hrte.}%
	\item {Implementierte parallele Designs in Verilog HDL unter Verwendung von Vivado und Open-Source-IPs auf Zynq-7000 ZC702 und Digilent Nexys4 DDR FPGA-Boards.}%
%	\item {Konzipierte eine neue parallele Algorithm f{\"u}r die Dekomposition der Cholesky-Matrix; Verbesserung der Zeitkomplexit{\"a}t von $\mathcal{O} \left( N^3 \right)$ zu $\mathcal{O} \left( N \right)$.}%
\end{cvInnerItems}%
}

\cvSchoolFlush
%---------------------------------------------------------

% \cvschool
% 	{High School} % Degree
% 	{Class XII Maharashtra Higher Secondary Certificate} % Degree name
% 	{Deogiri College} % Institution
% 	{Aurangabad, India \ifbool{showEmojis} {\space\emoji{flag-india}} {} } % Location
% 	{May 2013} % Date(s)
% 	{82.33\%} % Score
% 	{} % Thesis title, optional
% 	{} % Relevant coursework or other comments, optional
% %---------------------------------------------------------

% \cvschool
% 	{Secondary School}
% 	{Class X Central Board of Secondary Education} % Degree name
% 	{Nath Valley School} % Institution
% 	{Aurangabad, India \ifbool{showEmojis} {\space\emoji{flag-india}} {} } % Location
% 	{2011} % Date(s)
% 	{GPA 10 / 10} % Score,
%	{} % Thesis title, optional
%	{} % Relevant coursework or other comments, optional
% %---------------------------------------------------------

\end{cveducation}
