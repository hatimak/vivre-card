% IMPORTANT: THIS TEMPLATE NEEDS TO BE COMPILED WITH XeLaTeX
%
% This template uses several fonts not included with Windows/Linux by
% default. If you get compilation errors saying a font is missing, find the line
% on which the font is used and either change it to a font included with your
% operating system or comment the line out to use the default font.
%-------------------------------------------------------------------------------
% Known Issues:
% 1. Overflows onto second page if any column's contents are more than the
% vertical limit (http://www.sascha-frank.com/latex-minipage.html - possible
% solution?)
% 2. Hacky space on the first bullet point on the second column.
%-------------------------------------------------------------------------------

\documentclass[]{deedy}
\begin{document}
%
%-------------------------------------------------------------------------------
% Title Name
%-------------------------------------------------------------------------------
\namesection{Hatim}{Kanchwala}{Pre-final Year \textbullet{} Undergraduate \textbullet{} Electrical Engineering \textbullet{} Indian Institute of Technology Patna}{{\faMapMarker} A-437, Boys' Hostel, IIT Patna, Amhara Village, Bihta, Bihar, India - 801118}{{\faMobile} (+91) 966 5154 719 \textbullet{} {\faEnvelopeO} \href{mailto:hatim@hatimak.me}{hatim@hatimak.me}, \href{mailto:hatim.ee14@iitp.ac.in}{hatim.ee14@iitp.ac.in} \textbullet{} {\faGlobe} \href{http://hatimak.me}{hatimak.me} \textbullet{} {\faGithub} \href{https://github.com/hatimak}{hatimak} \textbullet{} {\faLinkedinSquare} \href{https://www.linkedin.com/in/hatimak/}{hatimak}}
%
%-------------------------------------------------------------------------------
% Column One
%-------------------------------------------------------------------------------
\begin{minipage}[t]{0.27\textwidth}
\begin{flushleft}
%
%-------------------------------------------------------------------------------
% Education
%-------------------------------------------------------------------------------
\section{Education}
%
\runsubsection{Indian Institute of Technology Patna\\}
\descript{B. Tech. in Electrical Engineering}
\location{{\faCalendar} 2014 --- 2018}
\location{{\faMapMarker} Bihta (BR), India}
CPI: 7.59 / 10.0 (upto 5\textsuperscript{th} semester)\\
\sectionsep
%-------------------------------------------------------------------------------
%
\runsubsection{Deogiri College\\}
\descript{Higher Secondary}
\location{{\faCalendar} 2011 --- 2013}
\location{{\faMapMarker} Aurangabad (MH), India}
\sectionsep
%-------------------------------------------------------------------------------
%
\runsubsection{Nath Valley School\\}
\descript{Primary \& Secondary}
\location{{\faCalendar} 2001 --- 2011}
\location{{\faMapMarker} Aurangabad (MH), India}
\sectionsep
%-------------------------------------------------------------------------------
%-------------------------------------------------------------------------------
% Skills
%-------------------------------------------------------------------------------
\section{Skills}
\subsection{Programming}
C/C++ \textbullet{} Assembly \textbullet{} Verilog \textbullet{} JavaScript \textbullet{} Python \textbullet{} Java \textbullet{} Scala \textbullet{} Shell \textbullet{} HTML/CSS \textbullet{}\\
\LaTeX\ 
%
\subsection{Software}
GNU/Linux \textbullet{} git/GitHub \textbullet{} MATLAB \textbullet{} Simulink \textbullet{} Xilinx ISE \textbullet{} Multisim \textbullet{} Synopsys \textbullet{} Pyxis \textbullet{} MPLAB IDE \textbullet{} Proteus \textbullet{} NumPy/SciPy \textbullet{} Eclipse IDE
%
\subsection{Hardware}
Xilinx Spartan FPGA \textbullet{} PIC Microcontroller \textbullet{} TMS320 DSP Chip \textbullet{} Arduino \textbullet{} 8051 Microcontroller \textbullet{} Raspberry\\
Pi \textbullet{} Teensy
%
\subsection{Languages}
English \textbullet{} Hindi \textbullet{} Gujarati \textbullet{}\\
Urdu \textbullet{} Marathi \textbullet{} Arabic
\sectionsep
%-------------------------------------------------------------------------------
%
\end{flushleft}
\end{minipage}
\hfill
%
%-------------------------------------------------------------------------------
% Column Two
%-------------------------------------------------------------------------------
\begin{minipage}[t]{0.69\textwidth}
\begin{flushleft}
%
%-------------------------------------------------------------------------------
% Interests
%-------------------------------------------------------------------------------
\section{Interests}
Computer Organisation \& Architecture \textbullet{} Embedded Systems Design \textbullet{} Adaptive\\
Filters \textbullet{} FPGA based Digital Signal Processing
\sectionsep
%-------------------------------------------------------------------------------
%-------------------------------------------------------------------------------
% Experience
%-------------------------------------------------------------------------------
\section{Experience}
%
\runsubsection{Aficionado Ventures}
\vspace{1.1pt}
\descript{\textbullet{} Backend Engineer}
\location{{\faTag} Startup \textbullet{} {\faCalendarCheckO} FEB. 2016 --- OCT. 2016 \textbullet{} {\faMapMarker} Gurugram (HR), India}
%\vspace{\topsep}% Hacky fix for awkward extra vertical space
\begin{tightemize}
\item Purchasing platform helping restaurants procure better quality produce from competitive collection of vendors.
\item Designed reactive backend architecture - MeteorJS server, MongoDB database, Heroku/mLab web app hosting, and Cordova for cross-platform mobile apps.
\item Implemented MVP-stage marketplace platform with bilingual search to look up products in English/Hinglish. Prototyped conversational UX via messaging bot.
\item Started work on vendor inventory and restaurant demand prediction models.
\end{tightemize}
\sectionsep
%-------------------------------------------------------------------------------
%
\runsubsection{Enhance Flashrom with features to read \& write multiple status registers and lock \& unlock memory space}
\vspace{1.1pt}
\descript{\textbullet{} \href{https://summerofcode.withgoogle.com/archive/2016/projects/5439533130711040/}{Google Summer of Code 2016 student with Coreboot}}
\location{{\faTag} Open-source \textbullet{} {\faCalendarCheckO} FEB. 2016 --- AUG. 2016 \textbullet{} {\faGithub} \href{https://github.com/hatimak/flashrom}{hatimak/flashrom}}
\begin{tightemize}
\item Designed multiple status registers model to abstract chip diversities across manufacturers into single consistent interface.
\item Developed rutines to lock/unlock memory space governed by bits in status register(s), handle configuration bits, and automatically generate BP range table for some chips.
\item Added functionality to access/lock OTP memory areas.
\item Developed CLI to expose new infrastructure. Tested on physical GigaDevice SPI chips using Raspberry Pi (over SPI bus), and Teensy.
\end{tightemize}
\sectionsep
%-------------------------------------------------------------------------------
%
\runsubsection{Weave}
\vspace{1.1pt}
\descript{\textbullet{} Co-founder \& Product Developer}
\location{{\faTag} Startup \textbullet{} {\faCalendarCheckO} JUNE 2015 --- DEC. 2015 \textbullet{} {\faMapMarker} Patna (BR), India}
\begin{tightemize}
\item Implemented prototype in JavaScript to build NURBS model of user, using input features and weighted combination of precomputed basis models.
\item Developed algorithm to emulate dyanmics for clothes on NURBS model - users can try clothes virtually.
\item Despite garnering client \& investor interest, startup failed because product-market fit was not right.
\end{tightemize}
\sectionsep
%-------------------------------------------------------------------------------
%-------------------------------------------------------------------------------
% Projects
%-------------------------------------------------------------------------------
\section{Projects}
%
\runsubsection{FPGA implementation of NLMS adaptive filtering algorithm for signal enhancement\\}
\vspace{1.1pt}
\location{{\faTag} Research \textbullet{} {\faCalendar} FEB. 2017 --- PRESENT \textbullet{} {\faGithub} \href{https://github.com/hatimak/zephyr}{hatimak/zephyr}}
\location{Adviser: Dr Yatendra Kumar Singh}
\begin{tightemize}
\item Implementing Normalized Least Mean Squares (NLMS) adaptive filtering algorithm to extract desired audio from signal corrupted with additive autoregressive noise.
\item Prototyping on Xilinx Spartan-3E FPGA in Verilog HDL, with focus towards optimised placing and routing.
\item Investigating performance gain by FPGA over typical DSP chip (TMS320).
\end{tightemize}
\sectionsep
%-------------------------------------------------------------------------------
%
\end{flushleft}
\end{minipage}
%-------------------------------------------------------------------------------
%
\pagebreak
%
%-------------------------------------------------------------------------------
% PAGE 2
%-------------------------------------------------------------------------------
% Column One
%-------------------------------------------------------------------------------
\begin{minipage}[t]{0.27\textwidth}
\begin{flushleft}
%
%-------------------------------------------------------------------------------
% Coursework
%-------------------------------------------------------------------------------
\section{Coursework}
%
\subsection{Undergraduate}
Embedded Systems\\
\vspace{1.5pt}
VLSI Design\\
\vspace{1.5pt}
Digital Electronics \& Microprocessors\\
\vspace{1.5pt}
Semiconductor Devices \& Circuits\\
\vspace{1.5pt}
Analog Integrated Circuits\\
\vspace{1.5pt}
Digital Signal Processing\\
\vspace{1.5pt}
Control Systems\\
\vspace{1.5pt}
Communication Systems\\
\vspace{1.5pt}
Electromagnetic Theory\\
\vspace{1.5pt}
Electronic Instrumentation\\
\vspace{1.5pt}
Electrical Power Systems\\
\vspace{1.5pt}
Electrical Machines\\
\vspace{1.5pt}
Linear Algebra\\
\vspace{1.5pt}
Probability \& Random Processes\\
\vspace{1.5pt}
Algorithms \& Data Structures
%
\subsection{MOOC}
The Hardware/Software Interface\\
\vspace{1.5pt}
\href{https://www.coursera.org/account/accomplishments/certificate/7V2KZSKAL7ZJ}{Machine Learning}\\
\vspace{1.5pt}
Game Theory\\
\vspace{1.5pt}
\href{https://www.coursera.org/account/accomplishments/certificate/98ZD8GFQ8BMS}{Valuation: Risk \& Return}\\
\vspace{1.5pt}
\href{https://www.coursera.org/account/accomplishments/certificate/P637RR9C8B7T}{Valuation: Time Value of Money}
\sectionsep
%-------------------------------------------------------------------------------
%
\end{flushleft}
\end{minipage}
\hfill
%
%-------------------------------------------------------------------------------
% Column Two
%-------------------------------------------------------------------------------
\begin{minipage}[t]{0.69\textwidth}
\begin{flushleft}
%
%-------------------------------------------------------------------------------
% Projects (continuation)
%-------------------------------------------------------------------------------
\vspace{5pt} %Hacky fix to align top text on page 2
\runsubsection{Biometric attendance system suitable for economic and low-power limited-connectivity remote deployment\\}
\vspace{1.1pt}
\location{{\faCalendarCheckO} JAN. 2015 --- OCT. 2015}
\begin{tightemize}
\item Developed code to interface 8051 (Atmel AT89S51) with fingerprint reader (R305) and GSM/GPRS module (SIM900A) over serial port via multiplexer.
\item Developed human interface using 16$\times$2 LCD and keypad.
\item Post biometric authentication, attendance data is transmitted by SIM900A via SMS to server. Fingerprints stored locally only.
\item Developed firmware in C and ASM. Microcontroller operations simulated in Proteus.
\end{tightemize}
\sectionsep
%-------------------------------------------------------------------------------
%-------------------------------------------------------------------------------
% Mini Projects
%-------------------------------------------------------------------------------
\section{Mini Projects}
%
\runsubsection{8085 Instruction Set Architecture prototype on FPGA with basic pipelining\\}
\vspace{1.1pt}
\location{{\faCalendar} FEB. 2017 --- PRESENT \textbullet{} {\faGithub} \href{https://github.com/hatimak/marineford}{hatimak/marineford}}
\location{Adviser: Dr Kailash Chandra Ray}
\begin{tightemize}
\item Implementing Intel 8085 microprocessor ISA on Xilinx Spartan-3E FPGA in Verilog HDL. Developing ASM testbenches to profile performance delta due to pipelining.
\end{tightemize}
\sectionsep
%-------------------------------------------------------------------------------
%
\runsubsection{Implementation of Viola-Jones object detection framework\\}
\vspace{1.1pt}
\location{{\faCalendar} JAN. 2017 --- PRESENT \textbullet{} {\faGithub} \href{https://github.com/hatimak/vj-goggles}{hatimak/vj-goggles}}
\location{Adviser: Dr Mahesh H. Kolekar}
\begin{tightemize}
\item Implementing algorithm proposed in \href{https://www.cs.cmu.edu/~efros/courses/LBMV07/Papers/viola-cvpr-01.pdf}{paper by P. Viola and M. Jones} in MATLAB and on TMS320 DSP chip.
\end{tightemize}
\sectionsep
%-------------------------------------------------------------------------------
%
\runsubsection{Full-custom design of ring oscillator\\}
\vspace{1.1pt}
\location{{\faCalendarCheckO} NOV. 2016 \textbullet{} {\faGithub} \href{https://github.com/hatimak/ring-osc}{hatimak/ring-osc}}
\location{Adviser: Dr Kailash Chandra Ray}
\begin{tightemize}
\item Designed core layout of ring oscillator using AMI05 (\SI{0.5}{\micro\metre}) CMOS Technology in Pyxis (Mentor Graphics), and verified output of back annotated simulation.
\end{tightemize}
\sectionsep
%-------------------------------------------------------------------------------
%
%-------------------------------------------------------------------------------
% Positions
%-------------------------------------------------------------------------------
\section{Positions}
%
\runsubsection{Coordinator, Startup Relations}
\vspace{1.1pt}
\descript{\textbullet{} Entrepreneurship Club, IIT Patna}
\location{{\faCalendar} APR. 2016 --- PRESENT}
\begin{tightemize}
\item Serving as on-campus mentor to early stage startups, helping them develop business plan, choose investor strategy, and network with advisers.
\item Building connections with startups and investors with vision to form investor panel for on-campus startups.
\item Delivered presentations as part of In-house Lecture series conducted by E-Club.
\end{tightemize}
\sectionsep
%-------------------------------------------------------------------------------
%
\runsubsection{Task Manager, Startup Relations}
\vspace{1.1pt}
\descript{\textbullet{} Entrepreneurship Club, IIT Patna}
\location{{\faCalendarCheckO} APR. 2015 --- APR. 2016}
\begin{tightemize}
\item Part of Organising Committee for E-Club’s flagship annual event, E-Week. Organised guest talks, workshops and pitching event.
\end{tightemize}
\sectionsep
%-------------------------------------------------------------------------------
%
\runsubsection{Volunteer, Startup Relations}
\vspace{1.1pt}
\descript{\textbullet{} Entrepreneurship Club, IIT Patna}
\location{{\faCalendarCheckO} AUG. 2014 --- APR. 2015}
\sectionsep
%-------------------------------------------------------------------------------
%
\end{flushleft}
\end{minipage}
%-------------------------------------------------------------------------------
\end{document}
\documentclass[]{article}